\begin{comment}
The emergence of Bitcoin in 2008 introduced a unique and innovative concept of decentralized digital currency. Bitcoin, as described in its white paper published on October 31st, 2008, proposed a "peer-to-peer electronic cash system" \cite{nakamoto2008bitcoin}. Since then, Bitcoin has experienced remarkable growth, attracting a significant number of users, businesses, and services into its ecosystem. As the Bitcoin network expanded, one of its fundamental components, mining, played a crucial role in securing and maintaining the integrity of the system.\\
To understand the significance of mining in the Bitcoin ecosystem, it is essential to gain a comprehensive understanding of Bitcoin itself. Chapter 2 provides an overview of Bitcoin, its underlying technology, and the composition of its network. We explore the concept of Bitcoin network nodes, their types, and roles in supporting the decentralized nature of the system. Additionally, we examine the extended Bitcoin network and its geographical distribution, offering insights into the global adoption and impact of Bitcoin.\\
Bitcoin's security relies on a consensus mechanism known as Proof of Work (PoW). In Chapter 2, we delve into the concept of PoW and its role in the mining process. PoW ensures that miners invest computational power to solve complex mathematical puzzles, thereby validating transactions and adding new blocks to the blockchain. Understanding PoW is vital for comprehending the motivations behind mining and its evolution over time.\\
Chapter 3 focuses on the mining process, exploring its mechanics, historical development, and evolution. We explain how mining works, the computational challenges involved, and the incentives for miners to participate in the network. By examining the evolution of mining, we highlight the transition from solo mining to pooled mining, which brought significant changes to the mining landscape. Pooled mining enabled miners to collaborate and share resources, improving their chances of receiving rewards in a more consistent and predictable manner.\\
Chapter 4 provides an in-depth exploration of the history of pooled mining protocols. We begin by examining the initial protocol, Getwork, its functionalities, and its limitations. Subsequently, we discuss the introduction of Getblocktemplate (GBT) and its improvements over Getwork. However, the central focus of this chapter is on Stratum (V1), which revolutionized pooled mining by introducing a more efficient and flexible protocol. We discuss the vulnerabilities and security issues associated with Stratum (V1) and the need for its subsequent update.\\
Chapter 5 delves into Stratum V2, a significant upgrade to the Stratum protocol. We explore the motivations behind the development of Stratum V2 and its inner workings. This chapter highlights the differences between Stratum V1 and Stratum V2, emphasizing the enhanced security, transaction selection power, and other improvements introduced by the new protocol. Moreover, we examine the current implementations of Stratum V2, providing insights into its adoption within the mining community.\\
To facilitate a practical understanding and experimentation with Stratum V2, Chapter 6 introduces the Stratum Reference Implementation (SRI). We explain how SRI works, guide readers on getting started with it, and discuss potential future directions for research and development. Additionally, we explore concepts such as SRI Pool fallback, non-custodial pools, and the importance of a benchmarking suite for Stratum V2.\\
The research presented in this thesis aims to shed light on the profound significance of Stratum V2 as a transformative protocol within the domain of Bitcoin pooled mining. By examining the history, evolution, and current implementations of mining protocols, we provide a comprehensive understanding of the advancements made and their implications for the future. The concluding chapter reflects on the potential impact of Stratum V2 and introduces concepts like SV2 protocol benchmarking and non-custodial pools as promising pathways for future innovation and SV2 adoption.\\
Through this exploration, we hope to contribute to the ongoing development and evolution of Bitcoin mining practices, shaping the landscape of this critical component of the Bitcoin ecosystem in the years to come.
\end{comment}
The emergence of Bitcoin in 2008 introduced a unique and innovative concept of decentralized digital currency. Bitcoin, as described in its white paper published on October 31st, 2008, proposed a "peer-to-peer electronic cash system" \cite{nakamoto2008bitcoin}. Since then, Bitcoin has experienced remarkable growth, attracting a significant number of users, businesses, and services into its ecosystem. As the Bitcoin network expanded, mining became a crucial component for securing and maintaining the system's integrity.\\
To fully realize the importance of mining in the Bitcoin ecosystem, it's important to get a comprehensive understanding of Bitcoin itself. Chapter 2 provides an overview of Bitcoin, its underlying technology, and the composition of its network. The concept of Bitcoin network nodes, their types, and roles in supporting the decentralized nature of the system are explored. Bitcoin's security is based on a consensus mechanism known as Proof of Work (PoW). Chapter 2 dives into the concept of PoW and its role in the mining process. PoW ensures that miners invest computational power to solve a mathematical problem, validating transactions and adding new blocks to the blockchain. Understanding PoW is crucial for understanding the motivations behind mining and its evolution over time.\\
Chapter 3 focuses on the mining process, exploring its mechanics, historical development, and evolution. The workings of mining, the computational challenges involved, and the incentives for miners to participate in the network are explained. The transition from solo mining to pooled mining, which brought significant changes to the mining landscape, is highlighted. Pooled mining enabled miners to collaborate and share resources, improving their chances of receiving rewards in a more predictable way.\\
Chapter 4 provides a deep exploration of the history of pooled mining protocols. The initial protocol, Getwork, its functionalities, and limitations are analyzed. After that, the introduction of Getblocktemplate (GBT) and its improvements over Getwork protocol are discussed. However, the central focus of this chapter is on Stratum (V1), which revolutionized pooled mining by introducing a more efficient and simpler protocol, becoming the standard "de facto" of the pooled mining activities. Then, the vulnerabilities and security issues associated with Stratum (V1) and the need for its subsequent update are discussed.\\
Chapter 5 dives into the core topic of the entire thesis, Stratum V2, a significant upgrade to the Stratum protocol. The motivations behind the development of Stratum V2 and its inner workings are explored. The differences between Stratum V1 and Stratum V2 are highlighted, particularly the enhanced security, transaction selection responsibility, and other improvements introduced by the new protocol. Moreover, the current implementations of Stratum V2 and its adoption path within the mining community are examined.\\
To facilitate a practical understanding and experimentation with Stratum V2, Chapter 6 introduces the Stratum Reference Implementation (SRI). The workings of SRI, guidance on getting started with it, and potential future directions for research and development are explained. Concepts such as SRI Pool fallback, non-custodial pools, and the importance of a benchmarking suite for Stratum V2 are explored.\\
The research presented in this thesis aims to shed light on the profound significance of Stratum V2 as a transformative protocol within the domain of Bitcoin pooled mining. By examining the history, evolution, and current implementations of mining protocols, a comprehensive understanding of the advancements made and their implications for the future is provided.\\\\
The aim of this exploration is to make a meaningful contribution to the continuous development and advancement of Bitcoin mining methods, consequently influencing the future direction of this essential aspect of the Bitcoin ecosystem.
