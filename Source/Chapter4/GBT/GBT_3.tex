As mentioned in the last words of the previous section, \textit{getblocktemplate} wasn't the only mining protocol which was developed and proposed during 2012, since the discussed issues related to the getwork protocol touched lots of different people and businesses in the mining sector.\\
During the same year, the developer called \textit{slush} proposed his alternative to the \textit{getwork} protocol, which is called \textbf{Stratum}. 
Marek "slush" Palatinus was the founder of one of the first Bitcoin mining pools, called \textit{slushpool}, and he was worried about the performance issues related to the mining protocol used at that time. With his experience as a mining pool operator, he developed the above-mentioned protocol alternative, publishing the design details right after the announcement of \textit{getblocktemplate}.\\
For this reason, under the announcement of GBT on Bitcoin Talk Forum can be found a lot of discussions related to the comparison between GBT and Stratum, adorned by not a few mutual criticisms.
During the weeks which followed the first discussions, mining pools operators started to test both protocols, trying to establish the one to implement in their own businesses.\\\\
At the end, performances of the \textit{Stratum} protocol were better than \textit{getblocktemplate}'s, and the design was a way cleaner and easier to be understood and implemented by the mining pools operators of that time.
Thanks to its efficiency improvements,  and for other specific features which will be deeply discussed in the next chapter, \textit{Stratum} became the standard "de facto" of the pooled mining protocols.  

