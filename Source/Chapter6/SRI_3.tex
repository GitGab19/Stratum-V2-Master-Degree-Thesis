As described and analyzed in the previous section, Stratum Reference Implementation (SRI) has achieved a substantial progress during last year, delivering crucial features and roles like the \textbf{Translator Proxy} and the \textbf{Job Negotiation}. \\ While the first permits to use the SV2 protocol without changing the ASIC machine's firmware, the second is the critical feature that allows a real decentralization of the transactions selection power (which is now entirely in the hands of mining pools operators).\\
However, as already anticipated, some details about the messages involved into the Job Negotiation Protocol are material of discussion: firstly it will be renamed into \textbf{Job Declaration Protocol}, and the reasons for that will be explained in the next subsection about \textit{SRI Pool fallback}.\\
Besides of that, SRI developers group defined many future further enhancements of the SV2 protocol, which are already been studied, such as the implementation of some specific payment pool, necessary to build a non-custodial pool, and the development of a protocol benchmarking suite.
\subsection{SRI Pool fallback}
The SRI Pool fallback is a feature which is already in the SRI roadmap, and it will be a very crucial piece of the protocol.\\
Basically, once the last little changes about the \textbf{Job Declarator Protocol} will be done, a miner who aims to work with a setup like the previously analyzed Config D (\ref{configD}), or even better Config A (\ref{configA}), will be able to build its own block templates, extracting the most profitable Bitcoin transactions from its local Template Provider. At this point, the miner will have a Job Declarator Client who is in charge of declaring this own block template to the Job Declarator Server (JDS) which will be Pool-side.\\
At this stage, the Pool-side JDS can still refuse the block template proposed by the miner (for any reason, could also be for censorship imposed by States or governmental agencies), and if this will be the case, the Job Declarator Client will \textbf{automatically declare the same block template} (containing the same transactions set) to another JDS of another mining Pool, choosing from a customized pre-configured backup list.\\
In the very extreme case in which all the JDS of the backup Pools are refusing the block template proposed by the miner, it will automatically start to do \textbf{solo mining}, without the need of any manual intervention.\\
By doing in this way, any possible future attacks to the censorship-resistance of the entire network will be extremely disincentivized and ineffective.

\subsection{Non-custodial pools}
Another subject of research of the SRI group is related to the current centralized and trusted payout mechanism used by the actual mining pools.
As described in section \ref{sec:pooled_mining}, nowadays the addresses inserted into the coinbase output to get the block reward is the ones belonging to the mining pools operators. Then, accordingly to the shares submitted from every miner who joins the pool, this reward is split and sent to the miners, through normal asynchronous Bitcoin transactions. The concentration of the entire funds in a central entity exposes pooled mining operations to a significant risk. As the payout process is based on a trusted centralized third-party pool service, miners must place complete trust in the fairness of their payouts, without the ability to independently verify whether the pool is withholding a portion of their rewards, a practice known as \textbf{pool skimming} \cite{bitcointalkMiningPool}.
The most valuable solution to address this issue is based on implementing a payout scheme where miners directly collect the coinbase reward, without the need for a centralized pool to control their funds: in this way, it would be possible to operate a fully non-custodial pool.\\
In the past, some possible solutions emerged from the market, but the most promising one was called \textbf{P2Pool}, who was announced in this way: <<P2Pool is a decentralized pool that works by creating a P2P network of miner nodes. These nodes work on a chain of shares similar to Bitcoin's blockchain. Each node works on a block that includes payouts to the previous shares' owners and the node itself. There is no central point of failure, making it DoS resistant.>> \cite{bitcointalk1500P2pool}.
However, its payout scheme was based on locking funds to miners' individual addresses within the coinbase transaction outputs, leading to a significant increase in the size of the coinbase: for this reason it revealed to be a very inefficient solution.

\noindent Three developers from the SRI team, published a RFC containing their own new payout scheme for a non-custodial mining pool on the bitcoin dev list \cite{linuxfoundationbitcoindevPayout}. As stated into the document, <<Our scheme is introduced through the concept of a payment pool, where the participants are the miners of the mining pool. The presented payment pool scheme uses ANYPREVOUT\cite{anyprevoutBIP118SIGHASH_ANYPREVOUT}, does not rely on any off-chain technology and it is trustless, in the sense that a participant does not have to trust in collaboration of all other participants: a non-collaborating participant is automatically ejected from the payment pool and it is not a threat for accessibility of funds. Our study assumes the pool to be centralised, but it can be generalised to decentralised pools. Our payment pool scheme is meant to be a future extension of Stratum V2 mining protocol.>> \cite{googleRFCpayment_pools03}


\subsection{SRI benchmarking suite}
As already told in \ref{SRI_intro}, SRI development started some years ago. However, a major mining protocol update like the one proposed by the SRI is very sensitive, due to the ever growing importance of the mining operations of nowadays.\\
During last year the SRI work started to get some real encouraging feedbacks from the Bitcoin community, thanks to the last major updates and to the communication efforts done in the last months.\\

\noindent By the way, to encourage Stratum V2 wide adoption, the SRI developers group think that a complete evaluation and precise measurements of the enhancements brought by SV2 is needed. A benchmarking suite which is able to easily test and benchmark protocol performances in different mining scenarios, capable of comparing the current version of Stratum (V1) with SV2 is necessary. In this way, mining industry professionals and the broader market will be able to easily understand every possible configuration permitted by SRI, evaluating and measuring themselves the potential benefits in terms of efficiency and consequently, profitability.
The main purpose of benchmarking is to demonstrate, with precise measurements, all the performance improvements brought by SV2, pushing at this point its natural adoption by both miners and mining pools.
