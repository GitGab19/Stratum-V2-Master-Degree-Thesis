In conclusion, this thesis has provided a comprehensive exploration of the evolution of mining protocols in the Bitcoin ecosystem, with a specific focus on the transformative potential of Stratum V2. By examining the history, mechanics, and limitations of previous protocols such as Getwork and Getblocktemplate, as well as the current dominant protocol Stratum (V1), the need for a more efficient and secure protocol became apparent. The emergence of Bitcoin as a decentralized digital currency highlighted the importance of mining in maintaining the system's integrity and security. The concept of Proof of Work (PoW) was introduced as a consensus mechanism, ensuring that miners invest computational power to validate transactions and add new blocks to the blockchain. However, the transition from solo mining to pooled mining brought significant changes, enabling miners to collaborate and increase their chances of receiving rewards in a more predictable manner. Unfortunately, because of the way Stratum (V1) protocol works, this benefit is gained at the expense of a centralization about the selection of the transactions to be included in blocks.\\

\noindent The main argument of this thesis centers around the significance of Stratum V2 as a transformative protocol for Bitcoin pooled mining. Stratum V2 addresses the centralization concerns associated with Stratum (V1) and introduces enhanced security, operational efficiency, transaction selection decentralization, and other improvements. By decentralizing power and giving more control to individual miners, Stratum V2 aims to maintain the decentralized and uncensorable nature of the Bitcoin network.\\
While Stratum V2 presents a promising solution, its adoption in the mining community is still in progress. Real-world data and concrete evidence showcasing the efficiency improvements brought by the protocol update will be essential in encouraging individual miners to embrace Stratum V2. Continued research and development, as well as the creation of benchmarking suites and practical implementations like the Stratum Reference Implementation (SRI), will play a vital role in further advancing the protocol and its adoption. It is crucial to emphasize the dangers associated with centralization in the hands of a few mining operators. Stratum V2 offers a pathway to mitigate these risks and maintain the decentralized nature of the Bitcoin network. By fostering a collaborative and secure mining environment, Stratum V2 has the potential to shape the future direction of Bitcoin mining methods and ensure the continued development of the Bitcoin ecosystem.\\

\noindent In conclusion, this thesis contributes to the understanding of the advancements made in mining protocols and highlights the significance of Stratum V2 as a groundbreaking protocol within the Bitcoin ecosystem. By addressing centralization concerns and introducing improvements in security and efficiency, Stratum V2 builds the way for a more decentralized and resilient Bitcoin network. As further research and development take place, and as the benefits of Stratum V2 become more evident through real-world implementations and data, it is expected that the mining community will increasingly adopt this protocol, ultimately enhancing the overall strength and uncensorability of the entire Bitcoin network.