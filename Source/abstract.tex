Bitcoin is a distributed, <<peer-to-peer electronic cash system>>\cite{nakamoto2008bitcoin}.\\
Since the publication of its white paper, on 31st of October 2008, Bitcoin has experienced a very rapid evolution. Many aspects of the protocol has changed in the years, permitting it to keep up with the exponential growth in terms of users, services and companies involved into it.\\
The primary focus of the entire research centers on one of the fundamental pillars of the entire ecosystem: Bitcoin mining.
More specifically, the thesis investigates the details of the approach that has characterized mining operations since Bitcoin's early history: pooled mining.\\
The introductory chapters of this thesis provide a comprehensive understanding of Bitcoin, its network composition, and the concept of Proof of Work. The subsequent chapters focus on mining, exploring both solo and pooled mining approaches, as well as the history and evolution of mining operations.\\
During its growth, many protocols have been developed to manage the pooled mining operations, such as Getwork, Getblocktemplate (GBT), and Stratum (V1).\\ 
However, the central focus of the thesis is Stratum V2: its inception, inner details, and the significant differences it brings compared to its predecessor, Stratum (V1). Concepts like protocol security, binary framing, and the power of transactions selection are deeply discussed, to provide all the detailed context necessary to understand the importance of this protocol update.
The current implementations of Stratum V2 are also discussed to provide a practical perspective on its adoption.
In addition to this, the thesis introduces the Stratum Reference Implementation (SRI), a key tool for understanding and experimenting with Stratum V2. Detailed explanations are provided on how SRI works, how to get started with it, and potential future directions for research and development.\\
The research concludes with a reflection on the advancements made in Stratum V2 and their implications for the future of Bitcoin pooled mining. The concept of SV2 protocol benchmarking is introduced, along with the exploration of non-custodial pools as a promising pathway for future innovation.
Overall, this thesis aims to dig into the profound significance of Stratum V2 as a transformative protocol in the domain of Bitcoin pooled mining, and its potential to shape the landscape of mining practices in the years to come.